\documentclass[12pt, letter] {IEEEtran}
\usepackage{graphicx}
\usepackage{amsmath}
\usepackage{amssymb}
\usepackage{textcomp}
\usepackage{mathtools}
\usepackage{natbib}
\usepackage{pgfplots}
\usepackage{url}

\bibliographystyle{abbrvnat}
\setcitestyle{authoryear,open={((},close={))}}

	\title  {Literature Review - Panama Canal}
	\author{Authored by: Julian To}
	\date {November 13th, 2019}

\begin{document}
	
	\maketitle
	
\section{Abstract}
The Panama Canal is one of the most geopolitically important waterways in the world, cutting travel times by weeks and being a key link in the modern, globalized economy. As such, it should come as no surprise that it it is a well researched and well studied topic. Everything from its environmental impact, cost of usage and operation, economic impact, and its usage for defense have all been studied and theorized since the first explorers dreamed of the mythological "Northwest Passage", or a waterway that connected the Atlantic to Pacific Ocean. 
\\ \\
Nowadays, more than ever, the Panama Canal is one of the cornerstones of the global shipping industry, and as a byproduct, the global economy. This paper will seek to discuss the operation of the canal with the reader, and hopefully give them a greater understanding of how it works, and more importantly, how this affects their lives by synthesizing over a hundred years of theory crafting and research to best inform the reader.
\\
\\
Another issue that is hotly debated, is the costs of transiting the canal. According to official website, \url{https://tolls.panama-canal.com/examples.html}

Tolls are calculated by first charging a toll on the ship depending on its size, and then for the cargo. Considering the average toll payed is around \$54,000 US dollars, the cost is something that must be considered when transiting the canal.


\section{Review}
The first and foremost issue regarding the Panama Canal is the traffic it receives and the also he size restrictions it places on ships, otherwise known as "Panamax". Thompson, in the 

\textit{
	Hispanic Engineer and Information Technology}, asserts that the expansion of the Panama Canal is "A project that affects ports all over the world" \cite{source1} Her assertion makes sense, since the size of the Panama Canal restricts some of the largest ships from entering it, such as some oil supertankers and the American Nimitz-class Supercarriers.
She further claims that at a certain point, it becomes more economical to bypass the Panama entirely, and instead route through the Suez. In order to remain competitive, the Panama Canal must keep up with the ever growing size of cargo ships, which continue to push new boundaries.
\\ \\
Further adding to the pressure of the Canal is a planned project a few hundred miles to the north: the proposed Nicaraguan Canal. Seeing the economic benefits of controlling a major trade node, the Nicaraguan government has also sought to build a new trans-continental canal to link the two oceans. Similar to the Panama Canal, this project has been proposed over a hundred years ago as various journals from the Scientific American discuss its viability and feasibility. However, with the collapse of the HK Nicaragua Canal Development Investment Company, the proposed canal project seems a long way from breaking ground. 
\\ \\
It might even be a good thing that the Canal isn't built. In an article titled "Rethink the Nicaragua Canal", Authors Huerte-Perez, Meyer, and Alvarez assert that the proposed canal back in 2014 was an idea bound for failure and hardships for locals, claiming "The
potential impact of “Lake Atlanta” and the canal on forests, wetlands, and coastal ecosystems has been grossly
underexplored". \cite {source2} \\

%\newline

The Nicaragua Canal would undoubtedly be a much longer route than its southern neighbor and would cut through and potentially destroy whole ecosystems, not to mention the destruction or diversion of already present waterways. Although its construction would undoubtedly change the way the Panama Canal is used, experts appear to suggest that its probably best if the a more proper plan laid down first before mindlessly digging a canal through the earth. This becomes especially true when you consider the human cost of building the Panama Canal, which numbers in the thousands. However, with modern technology and techniques, most agree that such a project, given enough capital and willpower, could be successfully completed. \newline

Another equally important issue regarding the operation of the canal is the issue of tolls. Tolls are necessary to cover the costs of operation, but charge too high, and ships will not enter the canal. Furthermore, considering that the average toll payed to cross is \$54,000 US Dollars, the cost plays a very large factor in the routing of maritime traffic. \\ 
According to the official website of the Panama Canal, the cost of the toll is calculated based on the combination of the size of the ship, and the amount of cargo carried \cite{source6}. Because some of the largest container ships may be carrying tens of thousands of tons of goods, these costs can quickly skyrocket, making the canal slightly less attractive, despite the time savings. \newline

In fact, a news report made in 2016 by Mike Schuler details the crossing of MOL Benefactor, whose toll came out to a staggering \$829,468 US Dollars. \cite{source3}. \newline
Not only are the costs prohibitive, but traffic is a big issue as well. Look at any marine traffic map and you will almost always see a large cluster of ships crowding around the entrance to either side of the canal. As ships get larger and larger, it becomes much more difficult to traverse the canal as stated by Thompson \cite{source1}. Ships must often make the transit through the narrowest parts in daylight hours, or require land-based trains to pull the ships, further causing jams. Just imagine your everyday commute to work, but 100 times worse in scale. \newline  

An additional concern that comes up often is the concerns regarding the environment. Continued expansion of the canal places great strain on local water supplies and potentially ruins local ecosystems. Of paramount focus is Gatun Lake. A significant amount of water is lost per cycle to lower a ship back to sea level, especially magnified during the dry season. \newline

The addition of extra, and larger locks would necessitate drawing more water usage from an already strained water supply, most experts agree. However, there are several water basins presently in construction or already complete. These water saving basins are designed to capture some of the water that is used to raise and lower the ships and allow it to be reused, preserving the local water resources and helping to preserve the local ecosystems and wildlife. \newline

Even with all of the issues, there is still a need for skilled operators to safely guide vessels through the narrow passages. Tugboat captain Mauricio Perez testifies, "The fears and dangers remain, although the boats are going through". \cite{source5} The fact of the matter is, the Panama Canal is too important to stop operation even in a deficit of skilled labour. Tugboats and locomotives on land are still needed to guide the larger ships through the locks. Perez echoes the thoughts of many of his colleagues, stating "Sometimes the only thing we can do is pray". \newline

As ships get larger and larger, mistakes become all the more costly. An accident involving a ship of this size can mean over a hundred thousand tons of goods and cargo into the sea, not only spelling financial disaster, but potentially blockading the canal for several days or even weeks, drastically harming commerce. Furthermore, the larger size of ships means longer transit times, and that means longer queues. \\
Many worry this could reduce the attractiveness of using the canal, especially with the significant toll prices involved. A final issue brought up by some is the necessity of daylight to transit the canal. Of course, with larger ships, more light and precision is needed, which may not be available at the level needed at night. This effectively cuts the usage time of the canal in half, further reducing the economic viability, some say.\newline

Considering the fact that the Panama Canal accounts for around ten percent of the GDP of the country, an accident that forces the canal shut for any period of time could spell economic disaster for its population of four million. While it is no secret that Panama has a much more developed economy than some of its neighbors, it is no doubt that the Canal plays a major role in the country, and a closure of the canal would spell disaster. Therefore, maintaining and keeping the canal competitive is something that defines the country of Panama, much as how the Suez is to Egypt.  \newline

In conclusion, the canal plays an extraordinarily important role in the world, and the vast amount of research that has been poured into this topic reflects that. Every aspect of the canal has been discussed time and time again, and I doubt that will change any time soon. Many parts of the operation must work together in tandem to keep global trade flowing. 

                                                                                                                           

\section {Bibliography}
\begin{thebibliography}{100}
	\bibitem {source1} 	Thompson, Garland L. “CIVIL ENGINEERS WIDEN THE PANAMA CANAL: A PROJECT THAT AFFECTS PORTS ALL OVER THE WORLD.” Hispanic Engineer and Information Technology, vol. 27, no. 1, 2012, pp. 13–17. JSTOR, www.jstor.org/stable/43757950.
	
	\bibitem{source2}  Huete-Perez, JA; Meyer, A; Alvarez, PJ; "Rethink the Nicaragua Canal" AMER ASSOC ADVANCEMENT SCIENCE, 1200 NEW YORK AVE, NW, WASHINGTON, DC 20005 USA
	
	\bibitem {source3} Schuler, Mike. “Containership Pays Nearly \$1 Million Toll to Cross the Expanded Panama Canal.” GCaptain, 6 July 2016, gcaptain.com/containership-pays-nearly-1-million-toll-to-cross-the-expanded-panama-canal/.
	
	\bibitem {source4} https://www.focus-economics.com/countries/panama
	
	\bibitem {source5} https://learningenglish.voanews.com/a/expanded-panama-canal-still-facing-problems/3718472.html
	
	\bibitem {source6} https://tolls.panama-canal.com/examples.html
	
\end{thebibliography}


\end{document}